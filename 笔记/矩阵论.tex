\section{线性空间}
    \def{线性空间}{一个非空集合, 带有加法和数乘, 且满足下列条件,
        {
            \item 加法封闭 $x+y \in V$
            \item 数乘封闭 $k x \in V$
            \item 存在零元 $x+0=x$
            \item 存在负元 $x+(-x) = 0$
            \item $1x = x$
            \item 交换律 $x+y = y+x$
            \item 分配律 $(k+l)x = kx+lx$
            \item 加法结合律 $x+(y+z) = (x+y) +z$
            \item 数乘结合律 $k(Lx) = (kl)x$
        }
    }

    \bf{性质}:
        \def{线性组合}{$x = a_1 x_1 +a_2 x_2+ ... +a_m x_m = [x_1 ... x_n] [ a_1 = X a\\ \vdots \\ a_3 ], \text{且}\ x \in V$}
            \bf{性质}: 
                \bf{线性无关/线性相关}: $\nexists / \exists\ a \neq 0, let x = \sum_{i=1}^n a_i x_i = 0$ 

        \def{维数}{线性空间中, 线性无关向量组所含向量最大的个数.}

        \def{基}{一个向量组, 且 (1) 线性无关; (2) 线性空间中所有向量都是该向量组的线性组合. $\forall x \in V, and\ x = \sum_{i=1}^{Dim} a_i x_i$. 其中, $x_i$称\bf{基向量}, $a_i \in a$称\bf{坐标}.}
            \bf{性质}: 
                \bf{基变换}: 新旧基之间的变换矩阵. $Y = X C$
                    \bf{性质}: 基变换矩阵是非奇异矩阵.
                \bf{坐标变换}: $a_x = C a_y$.
                    \Proof{$v = X a_x = Y a_y = X C a_y \quad\Rightarrow\quad a_x = C a_y$}
        
    \bf{例子} {
        \item n维实/复空间 $\mathbb R^N = {(x_1,...,x_n)| x_i\in \mathbb R} \quad \mathbb C^N = {(x_1,...,x_n)| x_i\in \mathbb C}$
        \item 矩阵空间 $\mathbb R^{n \times n}$
        \item \def{内积空间}{定义了内积的线性空间. 实线性空间的称Euclid空间; 复线性空间的称Unitary空间.
                \def{内积}{内积满足
                    {
                    \item 交换律:$<x, y>=\overline{<y, x>}$
                    \item 分配律:$<x, y+z> = <x , y> + <x, z>$
                    \item 齐次性:$<k x, y> = k <x, y>$
                    \item 非负性:$<x,x> \ge 0$, 当且仅当 $x = 0, <x,x> = 0$
                    }
                }
            }
                \bf{性质}: 
                    \def{正交}{两个向量的内积为零. $<x,y> = 0$}
                        \bf{算法}:
                            \bf{Schmidt正交化}{
                                \[b_i = a_i - \sum_{k=1}^{i-1} \frac{<a_i,b_j>}{<b_j,b_j>}b_j\]
                            }

                    \def{度量矩阵}{}
                    \def{Cauchy-Бyнияковскнй不等式}{$|<x, y>| e |x|\ |y|$}
    }

    \def{线性子空间}{
        线性空间中的一个非空集合, 且对线性运算的封闭.
        {
        \item 加法封闭 $x,y\in V_1 ,\quad x+y \in V_1$
        \item 数乘封闭 $x \in V_1, k x \in V_1$
        }
    }
        \bf{性质}: {
        \item 线性子空间也是线性空间. \Proof{这是因为$V_1$是V的子集合,所以$V_1$中的向量不仅对线性空间V已定义的线性运算封闭,而且还满足相应的8条运算律.}
        \item 每个非零线性空间至少有两个子空间,(1)自身; (2)仅由零向量所构成的子集合, 称为零子空间.             
        \item $dim\ V_1 e dim\ V$
        }

        \bf{张成}: $span(\vec x_1,...,\vec x_n ) = {\sum_{i=1}^n a_i \vec_i}$ 

        \bf{子空间运算}
            \item \def{交}{} 
            \item \def{和}{$V_1 + V_2 = \{z | z=x+y, x \in V_1, y \in V_2\}$}
                \bf{性质}
                    \def{直和}{
                        不相交的两个子空间的和. $V_1 \oplus V_2$
                        $V_1 + V_2$为直和 $\Leftrightarrow V_1 \cap V_2$等于零空间.
                    }
                        \bf{性质}
                            $V_1 \oplus V_2$中的任一向量只能唯一地表示为子空间V1与V2中各自一个向量的和.

            \bf{性质}{
            \item 子空间的交、和, 也是子空间. $V_1, V_2 \subseteq V,\ V_1 \cap V_2 \subseteq V,\ V_1 + V_2 \subseteq V$
            \item $dim\ V_1 + dim\ V_2 = dim\ (V_1 + V_2) + dim\ (V_1 \cap V_2)$
            \item $U=V_1 + V_2, then\ \quad U=V_1 \oplus V_2 \quad \Leftrightarrow \quad dim\ U=dim\ (V_1+V_2)=dim\  V_1+dim\  V_2 $
            }
        
\section{线性变换}
    \def{变换}{线性空间到自身的映射, 且$\forall x \in V$都有唯一$y \in V$与之对应.}

    \def{线性变换}{线性空间的一种变换, 且满足$T(k x + l y) = k(T x) + l(T y)$}

        \bf{例子}{
        \item \def{恒等变换}{\[T x = x \quad ;(\forall x \in V)\]}
        \item \def{零变换}{\[T x = 0 \quad ;(\forall x \in V)\]}

        \item \def{正交变换}{
                在内积空间中, 保持任意向量的长度不变的一种变换. Euclid空间中称正交变换, Unitary空间中称Unitary变换.
                \[<x, x> = <T x, T x>\]
                \bf{正交矩阵}:
                    \[A A^T = I\]
                    \[A A^H = I\]
            }

        \item \def{对称变换}{
                一种变换. Euclid空间中称对称变换, Unitary空间中称Hermite变换.
                \[<T x, y> = <x, T y>\]
                \bf{对称矩阵}:
                    \[A^T = A\]
                    \[A^H = A\]
            }

        \item \def{投影变换}{
                令线性空间分为不交的子空间L,M, 将线性空间沿M到L的投影的变换, 称投影变换.
                \bf{投影矩阵}:
                    \[P_{L,M} = [X & 0]\ [X & Y]^{-1}\]
            }
                \bf{性质}
                    \item P是投影矩阵 $\Rightarrow$ P是幂等矩阵($P = P^2$)
                    \item \[N(A) = R(I - A)\]

                \def{正交投影变换}{
                    设线性空间的子空间L, 将线性空间沿$L_\bot$到L的投影的变换, 称投影变换.
                    \bf{正交投影矩阵}:
                        \[P_L = X(X^H X)^{-1}X^H\]
                }


        \item \def{初等旋转变换}{
            \[T_{ij} = I + \begin{pmatrix}0\\ & \cos\theta|_{(i,i)}& 0 & \sin\theta|_{(i,j)}& \\ & 0&0&0\\& \sin\theta|_{(j,i)}& 0 & \cos\theta|_{(j,j)}\\ &&&&0\end{pmatrix} - \begin{pmatrix}0\\ &1|_{(i,i)}\\&&0\\&&& 1 |_{(j,j)}\\& &&0\end{pmatrix}\]
            }
            \bf{性质} {
            \item $T$是正交矩阵.
            \item 设$x = (\xi_1, \cdots, \xi_{n})^T, \quad y= T_{ij}\ x=(\eta_1, \cdots, \eta_{n},)$, 则
                \[\{\begin{array}{l}
                \eta_i= c \xi_i+s \xi_j \\
                \eta_j=-s \xi_i+c \xi_j \\
                \eta_{k}=\xi_{k} \quad(k \neq i, j)
                \end{array}.\]
                且当$\xi_i^2+\xi_j^2 \neq 0$时, $c=\frac{\xi_i}{\sqrt{\xi_i^2+\xi_j^2}}, \quad s=\frac{\xi_j}{\sqrt{\xi_i^2+\xi_j^2}}$, 可使$\eta_i=\sqrt{\xi_i^2+\xi_j^2}>0, \eta_j=0$.
            }

        \item \def{初等反射变换}{$H x = (I-2 e_2 e_2^T) x$}
            \bf{性质} {
            \item 对称矩阵$H^T = H$, 正交矩阵$H^T H = I$, 对合$H^2 = I$, 自逆$H^{-1} = H$, $|H| = -1$.
            \item 初等旋转矩阵是两个初等反射矩阵的乘积.
            }
            \bf{算法} {
                \item \[u = \frac{b - |b| z}{|b - |b| z|}\]
                \item \[H = I - 2 u u^T\]
            }
        } 


        \bf{运算} {
        \item 加法
        \item 数乘
        \item 乘法
        \item 逆
        }

            
        \def{线性变换矩阵}{$T X = X A$ }
            \bf{性质}: 
                \item 运算 {
                    \item $(T_1 + T_2) X = X (A + B)$
                    \item $(k\ T_1) X = X (k\ A)$
                    \item $(T_1 T_2) X = X AB$
                    \item $T_1^{-1} X = X A^{-1}$
                    }
                \item 坐标变换: $b = A a$

        \bf{性质}
            \item \def{值域}{线性空间中, 所有向量在线性变换后的结果的集合, 也即线性变换后的线性空间. $R(T)=\{T x | x \in V\}$}
                \bf{性质}
                    \item 线性空间V的线性变换T的值域和核都是V的线性子空间.
                    \item \def{秩}{变换后的空间的维数, 也即值域的维数. $rank(A) = dim\ R(A) = dim\ R(A^T)$}

            \item \def{零空间}{线性空间中, 所有在线性变换为零向量的原向量的集合. $N(T) = \{x | T x = 0\}$}
                \bf{性质}
                    变换前线性空间维数 = 值域维数 + 零空间维数. \[n = dim\ R(A) + dim\ N(A)\]

            \item \def{不变子空间}{对于线性变换$T$的, 线性空间中的一个子空间, 且满足$\forall x \in V_1, T x \in V_1$}

            \def{特征值 \& 特征向量}{线性变换前后, 方向不改变的向量, 称特征向量$\lambda$; 特征向量在线性变换后长度变化的倍率, 称特征值$x$. \[T x = \lambda x\]}
                \bf{性质}
                    \bf{Hamilton-Cayley定理}{ 矩阵是其特征多项式的根.
                        \[
                            \varphi(\lambda) &=|\lambda I-A|=\lambda^n + a_1 \lambda^{n-1} + ... + a_{n-1} \lambda + a_n\\
                            \varphi(A) &=A^n + a_{1} A^{n-1}+ ... +a_{n-1} A + a_n I=0
                        \]
                    }
                    \def{最小多项式}{}

            \def{相似}{$\exists \text{非奇异矩阵}P, then \quad A \sim B \quad\Leftrightarrow\quad B = P^{-1} A P$}
                \bf{性质} {
                \item $A \sim A$ 
                \item $A \sim B \Leftrightarrow B \sim A$ 
                \item $A \sim B, B \sim C \Leftrightarrow A \sim C$
                \item 相似矩阵特征值、特征向量相同.
                \item 相似矩阵迹相同.
                }
                \bf{性质}
                    \def{对角矩阵}{$diag(\lambda_1, ... ,\lambda_n)$}
                        \bf{性质}
                            n阶矩阵相似于对称矩阵 $\Leftrightarrow$ 矩阵有n个线性无关的特征向量.

                    \def{Jordan标准型}{
                        线性空间中一定存在一个基, 使得线性变换可以表达为Jordan标准型.
                        \[J = diag(J_1(\lambda_1), ... , J_s(\lambda_s))\]
                        Jordan块:
                        \[J_i(\lambda_i) = [
                            \lambda_i&1&\\
                            &\lambda_i&1\\
                            &&\ddots&1\\
                            &&&\lambda_i
                        ]\]
                    }
                        \bf{性质}
                            对角矩阵是特殊的Jordan标准型.

                        \bf{算法}{
                        \item 行列式因子$D_i$ 所有$|\lambda I - A|$的i阶行列式子式的最大公因数.
                        \item 不变因子$d_i = \frac{D_i}{D_{i-1}} \quad;(D_0 = 1)$
                        \item 初等因子$d_i = \sum_j (\lambda - \lambda_j)^{a_j} \quad\Rightarrow\quad \{(\lambda - \lambda_j)^{a_j} | \forall i,j\}$
                        \item Jordan块$J_i(\lambda_i)$, 矩阵大小n为初等因子$(\lambda - \lambda_i)^{b_i}$的幂$a_i$
                        \item Jordan标准型$J = diag(J_1(\lambda_1), ... , J_s(\lambda_s))$
                        }

    \def{广义逆}{
        满足以下方程的解,
        \[ \{\begin{array}{ll}
            A X A &= A \\
            X A X &= X \\
            (A X)^H &= A X\\
            (X A)^H &= A X
        \end{array}. \]

        \[A^+ = (A^H A)^{-1} A^H\]
        \[A^+ = A^H (A A^H)^{-1}\]
    }
        \bf{性质}
            \item \[rank(A) = rank(A^H A) = rank(A A^H)\]
                    

\section{范数}
    \def{范数}{}

        \def{向量范数}{一类函数, 且满足{
            \item 非负性, $||A|| \ge 0$, 当且仅当$A = 0, ||A|| = 0$
            \item 齐次性, $||k A|| = |k| ||A||$
            \item 三角不等式, $||A + B|| e ||A|| + ||B||$
        }}
            \bf{例子}{
            \item \bf{$p$-范数}: $||x||_{p}=(\sum_{i=1}^{n}|x_i|^p)^{1 / p}$
            \item \bf{$\infty$-范数}: $||x||_\infty = \max|x_i|$
            \item \bf{椭圆范数}: $||x||_{A}=(x^T A x)^{\frac{1}{2}}$
            }

        \def{矩阵范数}{一类函数, 且满足{
            \item 非负性, $||A|| \ge 0$, 当且仅当$A = 0, ||A|| = 0$
            \item 齐次性, $||k A|| = |k| ||A||$
            \item 三角不等式, $||A + B|| e ||A|| + ||B||$
            \item 相容性, $||A B|| e ||A||\ ||B||$
        }}
            \bf{例子}{
            \item $||A||_{m_1} = \sum_{i,j} |a_{ij}|$
            \item $||A||_{m_2} = (\sum_{i,j} a_{ij}^2)^{\frac{1}{2}}$
            \item $||A||_{m_\infty} = n·\max_{i,j}|a_{ij}|$
            \item \bf{列和范数}: $||A||_1      = \max_j \sum_i |a_{ij}|$
            \item \bf{行和范数}: $||A||_\infty = \max_i \sum_j |a_{ij}|$
            \item \bf{谱范数}:   $||A||_2 = \sqrt{\max\ \lambda_i} \quad ,(\lambda_i)$为$A^H A$特征值.
            }
        
        \bf{关系}
            \bf{矩阵范数 \& 向量范数相容}: $||A x||_V e ||A||_M ||x||_V$

\section{矩阵分解}
    \def{矩阵分解}{}

        \bf{例子} {
        \item \def{上下三角分解}{将矩阵A化成上三角矩阵R与下三角矩阵L的乘积.$A = L R$}
        \item \def{上下三角对角分解}{将矩阵A化成上三角矩阵R, 对角矩阵D, 下三角矩阵L的乘积.$A = L D R$}

        \item \def{正交三角分解}{将非奇异矩阵A化成正交矩阵Q与非奇异上三角矩阵R的乘积.$A = Q R$}
                \bf{算法}  {
                \item \bf{Schmidt正交化方法} {
                    \item $A = [a_1, ..., a_n]$
                    \item Schmidt正交化 \[b_i = a_i - \sum_{k=1}^{i-1} \frac{<a_i,b_j>}{<b_j,b_j>}b_j\]
                    \item 正交矩阵Q
                        \[Q = [ \frac{b_1}{|b_1|}, ... , \frac{b_n}{|b_n|} ]\]
                    \item 非奇异上三角矩阵R
                        \[R = [ |b_1|\\ & \ddots\\ && |b_{n}| ] [ 1 & k_{21} & ... & k_{n1} \\ & 1 & ... & k_{n2} \\& & \ddots & \vdots \\& & & 1 ] \quad; k_ij = \frac{<a_i,b_j>}{<b_j,b_j>}\]\
                    }

                \item \bf{初等旋转变换方法} 
                }

        \item \def{满秩分解}{将矩阵A化成F G的乘积.$A = F G$}
            \Proof{\[A=P^{-1} B= [F \ S ] [ G\\ 0 ] = F G\]}
        \item \def{奇异值分解}{}
                \bf{算法} {
                \item $A^T A$ 计算特征值 $\lambda$, 特征向量$x$
                \item $V = [ \frac{x_1}{|x_1|}, ... ,\frac{x_n}{|x_n|} ], \quad \Sigma = diag(\lambda_1, ... ,\lambda_n)$
                \item $U_1 = A V \Sigma^{-1}$, 计算正交矩阵$U$
                \item 分解结果 $A = U \Sigma V^T$
                }
        }



\section{矩阵分析}
    \bf{矩阵函数计算}
        \[P^{-1} A P = \lambda arrow f(A) = P \lambda P^{-1}\]
        \bf{Jordan标准型计算}
            Jordan块:
            \[[\begin{array}{cccc}
                f(\lambda_i) & \frac{1}{1 !} f'(\lambda_i) & \cdots & \frac{1}{(m_i-1) !} f^{(m_i-1)}(\lambda_i) \\
                & f(\lambda_i) & \ddots & \vdots \\
                & & \ddots & \frac{1}{1 !} f'(\lambda_i) \\
                & & & f(\lambda_i)
            \end{array}]\]




