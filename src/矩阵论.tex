* 线性空间
    \def{线性空间} 
        一个非空集合, 带有加法和数乘, 且满足下列条件,
        * 加法封闭 $x+y \in V$
        * 数乘封闭 $k x \in V$
        * 存在零元 $x+0=x$
        * 存在负元 $x+(-x) = 0$
        * $1x = x$
        * 交换律 $x+y = y+x$
        * 分配律 $(k+l)x = kx+lx$
        * 加法结合律 $x+(y+z) = (x+y) +z$
        * 数乘结合律 $k(Lx) = (kl)x$

    \Property
        \def{线性组合} $x = a_1 x_1 +a_2 x_2+ ... +a_m x_m = [x_1 ... x_n] [ a_1 = X a\\ \vdots \\ a_3 ], \text{且}\ x \in V$
            \Property
                \bf{线性无关/线性相关} $\nexists / \exists\ a \neq 0, let x = \sum_{i=1}^n a_i x_i = 0$ 

        \def{维数} 线性空间中, 线性无关向量组所含向量最大的个数.

        \def{基} 一个线性无关的向量组$X = [x_1, ... , x_n]$, 且线性空间中所有向量都是该向量组的线性组合. $\forall x \in V, and\ x = \sum_{i=1}^n a_i x_i$. 其中, $x_i$称\bf{基向量}, $a_i \in a$称\bf{坐标}.
            \Property 
                \bf{基变换} 新旧基之间的变换矩阵. $Y = X C$
                    \Property 基变换矩阵是非奇异矩阵.
                \bf{坐标变换} $a_x = C a_y$.
                    \Proof{$v = X a_x = Y a_y = X C a_y \Rightarrow a_x = C a_y$}
        
    \Example 
        * n维实/复空间 $\mathbb R^N = {(x_1,...,x_n)| x_i\in \mathbb R} \quad \mathbb C^N = {(x_1,...,x_n)| x_i\in \mathbb C}$
        * 矩阵空间 $\mathbb R^{n \times n}$
        * 内积空间
            \def{内积空间} 定义了内积的线性空间. 实线性空间的称Euclid空间; 复线性空间的称Unitary空间.
                \def{内积} 
                    内积满足
                    * 交换律:$<x, y>=\overline{<y, x>}$
                    * 分配律:$<x, y+z> = <x , y> + <x, z>$
                    * 齐次性:$<k x, y> = k <x, y>$
                    * 非负性:$<x,x> \ge 0$, 当且仅当 $x = 0, <x,x> = 0$

            \Property
                \def{正交} 两个向量的内积为零. $<x,y> = 0$
                    \bf{算法}:
                        \bf{Schmidt正交化}{
                            $b_i = a_i - \sum_{k=1}^{i-1} \frac{<a_i,b_j>}{<b_j,b_j>}b_j$

                \def{度量矩阵}
                \def{Cauchy-Бyнияковскнй不等式} $|<x, y>| e |x|\ |y|$

    * 线性子空间
        \def{线性子空间}{
            线性空间中的一个非空集合, 且对线性运算的封闭.
            * 加法封闭 $x,y\in V_1 ,\quad x+y \in V_1$
            * 数乘封闭 $x \in V_1, k x \in V_1$
        }

        \Property
            * 线性子空间也是线性空间. \Proof{这是因为$V_1$是V的子集合,所以$V_1$中的向量不仅对线性空间V已定义的线性运算封闭,而且还满足相应的8条运算律.}
            * 每个非零线性空间至少有两个子空间,(1)自身; (2)仅由零向量所构成的子集合, 称为零子空间.             
            * $dim\ V_1 e dim\ V$

        \bf{张成} $span(\vec x_1,...,\vec x_n ) = {\sum_{i=1}^n a_i \vec_i}$ 

        \bf{子空间运算}
            * \def{交} 
            * \def{和} $V_1 + V_2 = \{z | z=x+y, x \in V_1, y \in V_2\}$
                    \Property
                        \def{直和}{
                            不相交的两个子空间的和. $V_1 \oplus V_2$
                            $V_1 + V_2$为直和 $\Leftrightarrow V_1 \cap V_2$等于零空间.
                        }
                            \Property
                                $V_1 \oplus V_2$中的任一向量只能唯一地表示为子空间V1与V2中各自一个向量的和.
                \Property
                    * 子空间的交、和, 也是子空间. $V_1, V_2 \subseteq V,\ V_1 \cap V_2 \subseteq V,\ V_1 + V_2 \subseteq V$
                    * $dim\ V_1 + dim\ V_2 = dim\ (V_1 + V_2) + dim\ (V_1 \cap V_2)$
                    * $U=V_1 + V_2, then\ \quad U=V_1 \oplus V_2 \quad \Leftrightarrow \quad dim\ U=dim\ (V_1+V_2)=dim\  V_1+dim\  V_2 $
            
    * 线性变换
        \def{变换} 线性空间到自身的映射, 且$\forall x \in V$都有唯一$y \in V$与之对应.

        \def{线性变换} 线性空间的一种变换, 且满足$T(k x + l y) = k(T x) + l(T y)$

        \Property
            * \def{值域} 线性空间中, 所有向量在线性变换后的结果的集合, 也即线性变换后的线性空间. $R(T)=\{T x | x \in V\}$
                \Property
                    * 线性空间V的线性变换T的值域和核都是V的线性子空间.
                    * \def{秩} 变换后的空间的维数, 也即值域的维数. $rank(A) = dim\ R(A) = dim\ R(A^T)$

            * \def{零空间} 线性空间中, 所有在线性变换为零向量的原向量的集合. $N(T) = \{x | T x = 0\}$
                \Property
                    变换前线性空间维数 = 值域维数 + 零空间维数. $n = dim\ R(A) + dim\ N(A)$

            * \def{不变子空间} 对于线性变换$T$的, 线性空间中的一个子空间, 且满足$\forall x \in V_1, T x \in V_1$

            * 特征值 \& 特征向量
                \def{特征值 \& 特征向量} 线性变换前后, 方向不改变的向量, 称特征向量$\lambda$; 特征向量在线性变换后长度变化的倍率, 称特征值$x$. $T x = \lambda x$
                \Property
                    * \bf{Hamilton-Cayley定理}{ 矩阵是其特征多项式的根.
                        $
                            \varphi(\lambda) &= |\lambda I-A|=\lambda^n + a_1 \lambda^{n-1} + ... + a_{n-1} \lambda + a_n
                            \varphi(A) &= A^n + a_{1} A^{n-1}+ ... +a_{n-1} A + a_n I=0
                        $
                        }
                    * \def{最小多项式}
                            \Property
                            * 最小多项式唯一
                            * 最小多项式整除所有以A为根, 1为首项的多项式$\varphi(\lambda)$.
                            * 最小多项式一定是特征多项式的所有零点都相同的因子. 且$m(\lambda) = \frac{\varphi(\lambda)}{D_{n-1}(\lambda)}$
                            * 相似矩阵, 转置矩阵, 最小多项式相同

            * \def{线性变换矩阵} 线性变换在基X下的矩阵表示. $T X = X A$ 
                \Property
                    * 运算 {
                        * $(T_1 + T_2) X = X (A + B)$
                        * $(k\ T_1) X = X (k\ A)$
                        * $(T_1 T_2) X = X AB$
                        * $T_1^{-1} X = X A^{-1}$
                        }
                    * 不同基下线性变换矩阵的转换 $A_Y = C^{-1} A_X C \text{ ,其中 } Y = C X$
                        \Proof{
                        $
                            T Y = Y A_Y         \tag{定义} 
                            T X C = X C A_Y     \tag{ Y = X C }
                            X A_X C = X C A_Y   \tag{ T X = X A_X }
                            A_X C = C A_Y
                            A_Y = C^{-1} A_X C
                        $
                        }
                    * 坐标变换 $b = A a$
                    * \def{奇异矩阵} 满秩的方阵.
                    * \def{正定矩阵} $x^T A x > 0$, 则称A正定.

            * 相似
                \def{相似} $\exists \text{非奇异矩阵}P, then \quad A \sim B \quad\Leftrightarrow\quad B = P^{-1} A P$
                \Property 
                    * $A \sim A$ 
                    * $A \sim B \Leftrightarrow B \sim A$ 
                    * $A \sim B, B \sim C \Leftrightarrow A \sim C$
                    * 相似矩阵特征值、特征向量相同.
                    * 相似矩阵迹相同.

                * \def{对角矩阵} $diag(\lambda_1, ... ,\lambda_n)$
                    \Property
                        n阶矩阵相似于对称矩阵 $\Leftrightarrow$ 矩阵有n个线性无关的特征向量.

                * \def{Jordan标准型}{
                        线性空间中一定存在一个基, 使得线性变换可以表达为Jordan标准型.
                        $J = diag(J_1(\lambda_1), ... , J_s(\lambda_s))$
                        Jordan块:
                        $J_i(\lambda_i) = [
                            \lambda_i&1&\\
                            &\lambda_i&1\\
                            &&\ddots&1\\
                            &&&\lambda_i
                        ]$
                    }
                    \Property
                        * 对角矩阵是特殊的Jordan标准型.

                    \bf{算法}{
                    * 行列式因子$D_i$ 所有$|\lambda I - A|$的i阶行列式子式的最大公因数.
                    * 不变因子$d_i = \frac{D_i}{D_{i-1}} \quad;(D_0 = 1)$
                    * 初等因子$d_i = \sum_j (\lambda - \lambda_j)^{a_j} \Rightarrow \{(\lambda - \lambda_j)^{a_j} | \forall i,j\}$
                    * Jordan块$J_i(\lambda_i)$, 矩阵大小n为初等因子$(\lambda - \lambda_i)^{b_i}$的幂$a_i$
                    * Jordan标准型$J = diag(J_1(\lambda_1), ... , J_s(\lambda_s))$
                    }

                * 广义逆
                    \def{广义逆}{
                        满足以下方程的解,
                        $ \{
                            A X A &= A \\
                            X A X &= X \\
                            (A X)^H &= A X\\
                            (X A)^H &= A X
                        . $
            
                        列满秩 $A^+ = (A^H A)^{-1} A^H$
                        行满秩 $A^+ = A^H (A A^H)^{-1}$
                    }
                    \Property
                        * $rank(A) = rank(A^H A) = rank(A A^H)$
                        * 满秩分解算广义逆 $A^+ = G^H (F^H A G^H)^{-1} F^H$

        \Example 
            * \def{恒等变换} $T x = x \quad ;(\forall x \in V)$
            * \def{零变换} $T x = 0 \quad ;(\forall x \in V)$

            * \def{正交变换}
                    在内积空间中, 保持任意向量的长度不变的一种变换. Euclid空间中称正交变换, Unitary空间中称Unitary变换.
                    $<x, x> = <T x, T x>$
                    \bf{正交矩阵}:
                        $A A^T = I$
                        $A A^H = I$

            * \def{对称变换}
                    一种变换. Euclid空间中称对称变换, Unitary空间中称Hermite变换.
                    $<T x, y> = <x, T y>$
                    \bf{对称矩阵}:
                        $A^T = A$
                        $A^H = A$

            * 投影变换
                \def{投影变换}
                    令线性空间分为不交的子空间L,M, 将线性空间沿M到L的投影的变换, 称投影变换.
                    \bf{投影矩阵}: $P_{L,M} = [X & 0]\ [X & Y]^{-1}$

                \Property
                    * 对投影的结果再做投影结果不变, 即, P是投影矩阵 $\Rightarrow$ P是幂等矩阵($P = P^2$)
                    * $A = \U [\Sigma & 0\\ 0 & 0] \V^H \Rightarrow A^+ = \V [\Sigma^{-1} & 0\\ 0 & 0] \U^H$
                    * $N(A) = R(I - A)$
                        \Proof
                            $
                                \text{设} x \in R(I-A) \Rightarrow x = (I-A) y
                                \because A^2 = A \Rightarrow A(I - A) = 0
                                \therefore A x = A(I - A) y = 0
                                \therefore x \in N(A)
                                \therefore R(I-A) \subseteq N(A) \tag{(1)}

                                \therefore dim(I - A) \le dim\ N(A) = n - dim\ A
                                \because I = A + (I - A) \Rightarrow n \le dim\ A + dim (I - A) \Rightarrow dim(I - A) \ge n - dim\ A
                                \therefore dim(I - A) = n - dim\ A \tag{(2)}

                                \because (1), (2)
                                \therefore R(I - A) = N(A)
                            $

                * 正交投影变换
                    \def{正交投影变换}
                        设线性空间的子空间L, 将线性空间沿$L_\bot$到L的投影的变换, 称投影变换.
                        \bf{正交投影矩阵}: $P_L = X(X^H X)^{-1}X^H$
                        
                    \Property
                        * 正交投影矩阵, 既是幂等矩阵, 也是Hermite矩阵.

            * \def{初等旋转变换}
                    $T_{ij} = I + [ 0\\ & \cos\theta|_{(i,i)}& 0 & \sin\theta|_{(i,j)}& \\ & 0&0&0\\&-\sin\theta|_{(j,i)}& 0 & \cos\theta|_{(j,j)}\\ &&&&0 ] - [ 0\\ &1|_{(i,i)}\\&&0\\&&& 1 |_{(j,j)}\\&&&0 ]$

                \Property
                    * $T$是正交矩阵.
                    * 设$x = (\xi_1, \cdots, \xi_{n})^T, \quad y= T_{ij}\ x=(\eta_1, \cdots, \eta_{n},)$, 则
                        $\{
                            \eta_i= c \xi_i+s \xi_j \\
                            \eta_j=-s \xi_i+c \xi_j \\
                            \eta_{k}=\xi_{k} \quad(k \neq i, j)
                        .$
                        且当$\xi_i^2+\xi_j^2 \neq 0$时, $c=\frac{\xi_i}{\sqrt{\xi_i^2+\xi_j^2}}, \quad s=\frac{\xi_j}{\sqrt{\xi_i^2+\xi_j^2}}$, 可使$\eta_i=\sqrt{\xi_i^2+\xi_j^2}>0, \eta_j=0$.

            * \def{初等反射变换} $H x = (I-2 e_2 e_2^T) x$
                    
                \Property 
                    * 对称矩阵$H^T = H$, 正交矩阵$H^T H = I$, 对合$H^2 = I$, 自逆$H^{-1} = H$, $|H| = -1$.
                    * 初等旋转矩阵是两个初等反射矩阵的乘积.

                \bf{算法} 
                    * $u = \frac{b - |b| z}{|b - |b| z|}$
                    * $H = I - 2 u u^T$
                    

* 范数
    \def{范数}

    * 向量范数
        \def{向量范数}
            一类函数, 且满足
            * 非负性, $||A|| \ge 0$, 当且仅当$A = 0, ||A|| = 0$
            * 齐次性, $||k A|| = |k| ||A||$
            * 三角不等式, $||A + B|| e ||A|| + ||B||$

        \Example 
            * \bf{$p$-范数} $||x||_{p}=(\sum_{i=1}^{n}|x_i|^p)^{1 / p}$
            * \bf{$\infty$-范数} $||x||_\infty = \max|x_i|$
            * \bf{椭圆范数} $||x||_{A}=(x^T A x)^{\frac{1}{2}}$

    * 矩阵范数
        \def{矩阵范数}
            一类函数, 且满足
            * 非负性, $||A|| \ge 0$, 当且仅当$A = 0, ||A|| = 0$
            * 齐次性, $||k A|| = |k| ||A||$
            * 三角不等式, $||A + B|| e ||A|| + ||B||$
            * 相容性, $||A B|| e ||A||\ ||B||$

        \Example 
            * $||A||_{m_1} = \sum_{i,j} |a_{ij}|$
            * $||A||_{m_2} = (\sum_{i,j} a_{ij}^2)^{\frac{1}{2}}$
            * $||A||_{m_\infty} = n·\max_{i,j}|a_{ij}|$
            * \bf{列和范数} $||A||_1      = \max_j \sum_i |a_{ij}|$
            * \bf{行和范数} $||A||_\infty = \max_i \sum_j |a_{ij}|$
            * \bf{谱范数}   $||A||_2 = \sqrt{\max\ \lambda_i} \quad ,(\lambda_i)$为$A^H A$特征值.
        
    \bf{关系}
        \bf{矩阵范数 \& 向量范数相容} $||A x||_V \le ||A||_M ||x||_V$

* 矩阵分解
    \def{矩阵分解}

    \Example  
        * \def{上下三角分解} 将矩阵A化成上三角矩阵R与下三角矩阵L的乘积.$A = L R$
            \bf{算法} 
                $
                    L_1      = I + [0 &0&...&0\\ \frac{a_{i1}}{a_{11}} &0&...&0\\ \vdots \\ \frac{a_{n1}}{a_{11}}&0&...&0]
                    L_1^{-1} = I + [0 &0&...&0\\-\frac{a_{i1}}{a_{11}} &0&...&0\\ \vdots \\-\frac{a_{n1}}{a_{11}}&0&...&0]
                    L_1^{-1} A = A_1

                    \text{重复上面步骤, 直至将 $A_i$ 化为上三角矩阵}
                    
                    U = A_{n-1}
                    L = \prod L_i
                    A = L U
                $
        * \def{上下三角对角分解} 将矩阵A化成上三角矩阵R, 对角矩阵D, 下三角矩阵L的乘积.$A = L D R$
            \bf{算法} {先上下三角分解, 在将对角线非全1一侧矩阵的对角线元素提出(并改为1)为对角矩阵D, 即 A = L D R}
        * \def{对称三角分解} 将对称正定矩阵化成对称的两个上下三角矩阵. $A = G G^T$
            \bf{算法} 
                * 先上下三角分解 $A = L D U = L D L^T \tag{因为对称正定矩阵}$
                * 
                    $
                        A = L (\sqrt(D))^2 L^T
                            = (L \sqrt(D)) (\sqrt(D) L^T)
                            = (L \sqrt(D)) (L \sqrt(D))^T
                            = G G^T
                    $

        * \def{正交三角分解} 将非奇异矩阵A化成正交矩阵Q与非奇异上三角矩阵R的乘积.$A = Q R$
            \bf{算法}  
                * \bf{Schmidt正交化方法} 
                    * $A = [a_1, ..., a_n]$
                    * Schmidt正交化 $b_i = a_i - \sum_{k=1}^{i-1} \frac{<a_i,b_j>}{<b_j,b_j>}b_j$
                    * 
                        $
                            Q = [ \frac{b_1}{|b_1|}, ... , \frac{b_n}{|b_n|} ]
                            R = [ |b_1|\\ & \ddots\\ && |b_{n}| ] [ 1 & k_{21} & ... & k_{n1} \\ & 1 & ... & k_{n2} \\& & \ddots & \vdots \\& & & 1 ] \quad; k_ij = \frac{<a_i,b_j>}{<b_j,b_j>}
                            A = Q R
                        $
                * \bf{初等旋转变换方法} 
                    * 对第1列, 初等旋转变换使其变为 $T_i a_1 = [b_{11}, 0,...,0]$
                    * $T_i = \prod_{i=0}^{n-1} T_{i(n-1-j)}$
                    * 重复上面步骤, 直至将 $A_i$ 化为上三角矩阵
                    * 
                        $
                            R = A_{n-1}
                            Q = (\prod_{i=0}^{n-1} T_{n-1-i} )^T
                            A = Q R
                        $
                * \bf{初等反射变换} 
                    * 对第1列, 初等旋转变换使其变为 $H_i a_1 = [b_{11}, 0,...,0]$
                        $
                            u_i = \frac{b_i - |b_i|}{| b_i - |b_i| |}
                            H_i = I - 2 u u^T
                            A_{i+1} = H_i A_i
                        $
                    * 重复上面步骤, 直至将 $A_i$ 化为上三角矩阵
                    * 
                        $
                            R = A_{n-1}
                            Q = (\prod_{i=0}^{n-1} H_{n-1-i} )^T
                            A = Q R
                        $

        * \def{满秩分解} 将矩阵A化成F G的乘积.$A = F G$
            \Proof $A=P^{-1} B= [F & S ] [ G\\ 0 ] = F G$
            
            \bf{算法} 
                初等行变换$A \to [ G\\ 0 ]$, 取A左侧rank(A)列作为F, 则$A = F G$

        * \def{奇异值分解} 将矩阵A化成两个Unitary矩阵$U, V$, 和一个非零奇异值组成的矩阵$\Sigma$的乘积. $A = U [\Sigma & 0 \\ 0 & 0] V^T$

            \bf{算法} 
                * $A^T A$ 计算特征值 $\lambda$, 特征向量$x$
                * $V = [ \frac{x_1}{|x_1|}, ... ,\frac{x_n}{|x_n|} ], \quad \Sigma = diag(\sqrt(\lambda_1), ... ,\sqrt(\lambda_n))$
                * $U_1 = A V \Sigma^{-1}$, 计算正交矩阵$U$
                * $ A = U [\Sigma & 0 \\ 0 & 0] V^T $



* 矩阵分析
    * 矩阵函数
        \Property
            * 若 $AB = BA$, 则$e^A e^B = e^B e^A = e^(A+B)$
            *   $P^{-1} A P = \Lambda \Rightarrow f(A) = P f(\Lambda) P^{-1}$
                $P^{-1} A P = J \Rightarrow f(A) = P f(J) P^{-1}$
            * Jordan标准型计算:
                Jordan块:
                $ f(J_i) = [
                    f(\lambda_i) & \frac{1}{1 !} f'(\lambda_i) & \cdots & \frac{1}{(m_i-1) !} f^{(m_i-1)}(\lambda_i) \\
                    & f(\lambda_i) & \ddots & \vdots \\
                    & & \ddots & \frac{1}{1 !} f'(\lambda_i) \\
                    & & & f(\lambda_i)
                ]$


    * 矩阵微积分




