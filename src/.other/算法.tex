
* 数据结构

	* 线性表
		* 数组

		* 链表
			* 双向链表

		* 队列
			* 先进后出

		* 栈
			* 先进先出

	* 树
		* 二叉树

		* 红黑树
			\Property
				* Every node is either red or black.
				* The root is black.
				* Every leaf (NIL) is black.
				* If a node is red, then both its children are black.
				* For each node, all simple paths from the node to descendant leaves contain the same number of black nodes.

		* 堆
			\Property
				* h深的树, 元素个数为$\{ 2^h, 2^{h+1} - 1 \}$ 
					\Proof
						* 最多, 满树
						$n = \sum_{i=0}^h 2^i = \frac{1 - 2^{h+1}}{1 - 2} = 2^{h+1} - 1$
						* 最少, 底层余一
						$n = \sum_{i=0}^{h-1} 2^i + 1 = 2^h$

				* n元堆的高度为$⌊\lg n⌋$
					\Proof
						h深的树, 元素有$\{ 2^h, 2^{h+1} - 1 \}$
						. $∴ lg n \in [h, h + 1]$
						故下取整为$ h = ⌊\lg n⌋$

	* 图

	* Hash表


* 算法
	* 排序
		* 快速排序
			\Property
				* 时间复杂度: 上限n²,大概率平均值n log(n)
				*	[流程]: 
				*		* 最后元素作为参考元素[Ref],[border]分割比参考元素大或小的界限,指向大的最初段
						* [ai] 从第一元素开始,直至 Ref 前一元素位置
						*	1).若 ai >=Ref, 不处理
							2).若 ai < Ref, 
								ai 与 border元素交换位置, border++
						* 遍历结束后, 交换Ref 与 border
						* Ref 左右比他大/小的两方元素,分别从头开始. 
			
		* 堆排序

		* 选择排序

		* 冒泡排序

		* 归并排序

		* 希尔排序

	* 图论
		* 图

		* 最短路径
			* Dijkstra算法

			* Floyd算法
				[输入]:	* 图的邻接矩阵Graph
				[输出]: * 距离矩阵Distance	* 后继节点矩阵Path
					* 邻接矩阵Graph, 即.图的权值矩阵
					* 距离矩阵Distance: i到j最短路径长度
					* 后继节点矩阵Path: 记录两点间的最短路径, 表示从Vi到Vj需要经过的点
				\bf{原理}
					* Floyd的本质的动态规划, Dijkstra的本质是贪心
					* 状态转移方程:
						Distance[i,j] = min{ Distance[i,k] + Distance[k,j] , Distance[i,j] }

				\bf{流程}
					* 初始化距离矩阵Distance = 权值矩阵Graph
						初始化后继矩阵Path(i,j) = j
					* 对于每一对顶点 i 和 j, 看是否存在点 k 使得u->k->v比已知路径更短
							即. Distance[i,j] = min{ Distance[i,k] + Distance[k,j] , Distance[i,j] }
						若是,则更新Distance, Path(i,j) = k
				
			\Property
				* 时间复杂度:
					* Floyd	时间复杂度$O(V^3)$, 3个for循环嵌套; 空间复杂度$O(V^2)$, 2个矩阵.
					* Dijkstra 时间复杂度$O(V^2)$

				* 对比Dijkstra \& Floyd
					Dijkstra 一次只能算出给定两点间的最短路径. 
					Floyd 一次可以算出任意两点间的最短路径. 

		* 最小生成树
			* Prim
				[输入]: 图的邻接链表[Graph], 节点数量[n]
				[输出]: 最小生成树,每一条有向边的 起点[TreeU],终点[TreeV],总边数[TreeCur]
				[原理]: 按点贪心, 每次加入已搜索点集u的最短边(u,v),其中v不属于已搜索点集的点v

				\bf{流程}
					* 初始化[已搜点集 VertexNew]
					* 将第一个图的节点, 加入VertexNew
					* 开始迭代, 直至所有节点均已搜索完成, 即VertexNew已满
						* 在已搜点集, 寻找最短边(u,v), 其中u∈VertexNew, v ∉ VertexNew
						* 将边(u,v)加入最小生成树, 将v加入VertexNew

			* Kruskal
				\bf{流程}
					* 初始化未搜索边集EdgeNew = E0图边集
					* 开始迭代,直至未搜索边集为空集
						* 在边集合,选择最短边(u,v)
						* 若(u,v)不在同一颗树, u,v所在两棵树合并,(u,v)加入该树
						* 点集中删(u,v)
					* 最后剩下的那棵树,就是最小生成树

			\Property
				* 时间复杂度:
					* Prim 时间复杂度 $O(E·logV)$
					* Kruskal 时间复杂度 $O(E·logV)$
				* 对比Prim \& Kruskal
					Kruskal	是按边贪心,适合稀疏图. 
					Prim 是按点贪心,适合稠密图. 

		* 网络最大流
			* Dinic
				[原理]: 贪心 + "反悔"机制
				增广路: 就是源->汇的一条路径. 
					利用深搜DFS,找增广路. 
					利用广搜BFS,确定此时各顶点的层次. 
					利用添加反向边,协助"反悔". 
					广搜BFS是Dinic对于EK的优化. 
					广搜BFS,用队列queue. 
					深搜BFS,用递归or栈stack. 

		* 商旅问题
			\def{商旅问题} 遍历所有给定点的最短闭合路径.

	* 动态规划