* 概率论
	* 概率空间
		\def
			概率空间是一个三元组 $<\Omega, \mathcal F, \P>$
			- $\Omega$, 样本空间.
			- $\mathcal F$: 被选择的 $\Omega$的子集系列. 且满足
				- 包含空集、样本空间全集 $\emptyset, \Omega \in \mathcal F$
				- 取补封闭, 如果一个事件A在其中, 那么补集也需要在其中.  
					$A \in \mathcal F \Rightarrow A^C \in \mathcal F$
				- 可列并封闭 
					$A_1, A_2, \cdots \in \mathcal F \Rightarrow \bigcup_{i=1}^\infty A_i \in \mathcal F$
			- $\P : \mathcal F \to [0, 1]$ 概率测度, 且满足Kolmogorov公理.
	* Kolmogorov公理
		\def
			- 非负性 $\P(A) \in [0, 1] \quad ; \forall A \in F$
			- 规范性 $\P(\Omega) = 1$
			- 可列可加性 $\P (\bigcup_i A_i)=\sum_i \P(A_i)$
	* 概率
		\Theorem{全概率公式}
		\Theorem{Bayes公式}
			$P(A | B)=\frac{\P(B | A) \P(A)}{\P(B)}$
		\Theorem{大数定律}
			- 弱大数定理
				$\lim_{n \to \infty} \P\{|\frac{1}{n} \sum_{k=1}^n X_k-\mu|<\varepsilon\}=1$
			- Bernoulli大数定理
				$\lim_{n \to \infty} \P\{|\frac{f_A}{n}-p|<\varepsilon\} = 1$
		\Theorem{中心极限定律}
			$\lim_{n \to\infty} F_n(x) =\lim_{n \to\infty} \P\{\frac{\sum_{k=1}^n X_k - n \mu}{\sqrt{n} \sigma} ≤ x\}=\int_{-\infty}^x \frac{1}{\sqrt{2 π}} e^{-t^2 / 2} \d t=\Phi(x)$
	* 概率分布
		* 离散概率分布
			\Example
				* 0-1分布
					\def
						$\P\{X = k\} = p^k (1 - p)^k \quad; k \in \{0,1\}$
					\Property
						$
							\E(x) = p
							D(x) = p (1 - p)
						$
				* 二项分布
					\def
						$\P\{X = k\} = C^k_n p^k (1-p)^{n-k}$
					\Property
						$
							\E(x) = n p
							D(x) = n p(1 - p)
						$
				* 几何分布
					\def
						$\P\{X = k\} = p (1-p)^{n-1}$
						描述连续独立重复实验中, 首次成功所进行的实验次数.
					\Property
						$
							\E(x) = \frac{1}{p}
							D(x) = \frac{1 - p}{p^2}
						$
				* 超几何分布
					\def
						$\P\{X = k\} = \frac{C_M^k C_{N_M}^{n-k}}{C_N^n}$
					\Property
						$
							\E(x) = \frac{n M}{N}
							D(x) = \frac{n M}{N} (1 - \frac{M}{N})\frac{N - M}{N - 1}
						$
				* Poisson分布
					\def
						$\P\{X = k\} = \frac{λ^k}{k!} e^{-λ}$
					\Property
						$
							\E(x) = λ
							D(x) = λ
						$
					\Theorem{Poisson定理}
						$\lim_{n \to \infty, p \to 0, λ = n p} \frac{C_M^k C_{N_M}^{n-k}}{C_N^n} = \frac{λ^k}{k!} e^{-λ}$
		* 连续概率分布
			\Example
				* 均匀分布
					\def
						$
							f(x) = \{\mb \frac{1}{b-a} &\quad a < x < b \\ 0 &\quad other \me\right.
							F(x) = \{\mb 0 &\quad x < a \\ \frac{1}{b-a} &\quad a ≤ x < b \\ 1 &\quad b ≤ x \me\right.
						$
					\Property
						$
							\E(x) = \frac{a + b}{2}
							D(x) = \frac{(b - a)^2}{12}
						$
				* Normal分布
					\def
						$
							f(x) = \frac{1}{\sqrt{2 π \sigma^2}} e^{-\frac{(x - \mu)^2}{2 \sigma^2}} \quad; x \in (-\infty, +\infty)
						$
					\Property
						$
							\E(x) = \mu
							D(x) = \sigma^2
						$
					* 标准Normal分布
						$
							f(x) = \frac{1}{\sqrt{2 π }} e^{-\frac{x^2}{2}} \quad; x \in (-\infty, +\infty)
						$
					* 多元Normal分布
						\Property
							$D(x) = \.\Sigma$
							\Proof
								$
									F### &=\frac{1}{(2 π)^{\frac{D}{2}}} \frac{1}{|\.\Sigma|^{\frac{1}{2}}} \sum_{i=1}^D \sum_{j=1}^D \.u_i \.u_{j}^T \int e^{-\sum_{k=1}^D \frac{\.y_k^2}{2 λ_k}} y_i y_{j} \d \.y 
									&=\frac{1}{(2 π)^{\frac{D}{2}}} \frac{1}{|\.\Sigma|^{\frac{1}{2}}} \sum_{i=1}^D \.u_i \.u_i^T \int e^{-\sum_{k=1}^D \frac{y_k^2}{2 λ_k}} y_i^2 \d \.y  \tag{$i ≠ j, \.u_i \.u_j^T=0, \.u_i \.u_j$ 正交}
									&=\frac{1}{(2 π)^{\frac{D}{2}}} \frac{1}{|\.\Sigma|^{\frac{1}{2}}} \sum_{i=1}^D \.u_i \.u_i^T \int \prod_{k=1}^D e^{-\frac{y_k^2}{2 λ_k}} y_i^2 \d \.y 
									&=\frac{1}{(2 π)^{\frac{D}{2}}} \frac{1}{|\.\Sigma|^{\frac{1}{2}}} \sum_{i=1}^D \.u_i \.u_i^T(\int_{-\infty}^{+\infty} e^{-\frac{y_i^2}{2 λ_i}} y_i^2 \d y_i \times \prod_{k=1, k \neq i}^D \int_{-\infty}^{+\infty} e^{-\frac{y_X^2}{2 λ_k}} \d y_k)  \tag{积分乘法结合律}
									&=\frac{1}{(2 π)^{\frac{D}{2}}} \frac{1}{|\.\Sigma|^{\frac{1}{2}}} \sum_{i=1}^D \.u_i \.u_i^T((2 π λ_i)^{1 / 2} · λ_i \times \prod_{k=1, k \neq i}^D(2 π λ_k)^{1 / 2})  \tag{见下面推导}
									&=\frac{1}{(2 π)^{\frac{D}{2}}} \frac{1}{|\.\Sigma|^{\frac{1}{2}}}  · ((2 π)^{\frac{D}{2}} \prod_{k=1}^D λ_k) · (\sum_{i=1}^D \.u_i \.u_i^T λ_i) 
									&=\sum_{i=1}^D \.u_i \.u_i^T λ_i  \tag{\prod_{k=1}^D λ_k=|\Sigma|^{\frac{1}{2}}}
									&=\.\Sigma
								$
								当$k = i$时,
								$
									\int_{-\infty}^{+\infty} e^{-\frac{y_λ^2}{2 λ_i}} y_i^2 \d y_i &=(λ_i \sqrt{2 λ_i}) · \int_{-\infty}^{+\infty}(\frac{y_i^2}{2 λ_i})^{\frac{1}{2}} e^{-\frac{y_λ^2}{2 λ_i}} \d \frac{y_i^2}{2 λ_i} 
									&= (λ_i \sqrt{2 λ_i}) · 2 \Gamma(\frac{3}{2})  \tag{$\Gamma(z) = \int_0^{+\infty} x^{z-1} e^{-x} \d x$}
									&= \sqrt{2 π λ_i} · λ_i  \tag{$\Gamma(\frac{3}{2}) = \frac{\sqrt{π}}{2}$}
								$
								当$k ≠ i$时,
								$
									\int_{-\infty}^{+\infty} e^{-\frac{y_λ^2}{2 λ_k}} \d y_k &=(\sqrt{\frac{λ_k}{2}) · \int_{-\infty}^{+\infty}(\frac{y_k^2}{2 λ_k})^{-\frac{1}{2}} e^{-\frac{y_λ^2}{2 λ_k}} \d \frac{y_k^2}{2 λ_k} 
									&= (\sqrt{\frac{λ_k}{2}) · 2 \Gamma(\frac{1}{2})  \tag{$\Gamma(z) = \int_0^{+\infty} x^{z-1} e^{-x} \d x$}
									&= \sqrt{2 π λ_k}  \tag{$\Gamma(\frac{1}{2}) = \sqrt{π}$}
								$
				* $\Gamma$分布
					\def
						$
							f(x) = \{\mb \frac{1}{\beta^\alpha \Gamma(\alpha)} x^{a^{-1}} e^{-x / \beta} &\quad x \in (0, +\infty) \\ 0 &\quad x \in (-\infty, 0] \me\right.
						$
					\Property
						-
							$
								\E(x) = \alpha \beta
								D(x) = \alpha \beta^2
							$
						- 
							当$\alpha=1$时, $\Gamma$分布退化为指数分布;
							当$\alpha=n/2, \beta=\frac{1}{2}$时, $\Gamma$分布退化为$\chi^2$分布.
					* 指数分布
						\def
							$
								f(x) = \{\mb λ e^{-λ x} &\quad x \in (0, +\infty) \\ 0 &\quad x \in (-\infty, 0] \me\right.
								F(x) = \{\mb λ 1 - e^{-λ x} &\quad x \in (0, +\infty) \\ 0 &\quad x \in (-\infty, 0] \me\right.
							$
						\Property
							$
								\E(x) = \theta
								D(x) = \theta^2
							$
					* $\chi^2$分布
						\def
							$\frac{1}{2^{n / 2} \Gamma(n / 2)} x^{n / 2 - 1} e^{-x / 2}$
						\Property
							$
								\E(x) = n
								D(x) = 2 n
							$
* 随机过程
	* 平稳性 (时间平移不变性)
		\def
			$ 
				\P(x_{t_1}, ... , x_{t_n}) = \P(x_{t_1+\tau}, ..., x_{t_n+\tau}) \quad \forall \tau, t_1, ..., t_n \in \mathbb R, \quad n \in \mathbb N
			$
			统计特性不随随时间推移而改变. 
			\Note
				在时间序列上随意的任意间隔任意顺序的采样, 都具有时间平移不变性.
	* Markov性
