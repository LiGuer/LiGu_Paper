* 相对论力学
	* 相对性原理, 相互作用的传播速度
		\Theorem{相对性原理}
			所有自然定律, 在所有惯性参考系中都相同.
		\Theorem{相互作用的传播速度}
			相互作用的最大传播速度, 在所有惯性参考系中都一样(相对性原理可得). 可以证明, 这个速度是光在真空中的速度. (取$c\to \infty$, 即可过渡到经典力学.)
			$c = 2.998 × 10^8 m/s$ 
		\Property
			- 时间是相对的, 不同参考系中时间的流逝速度不同. "两个不同事件之间有一定的事件间隔"这句话, 只有在指定哪一个参考系下才有意义, 因为在参考系$K_1$同时发生的两个事件, 在参考系$K_2$可能是不同时的.
		\bf{实验}
			- 实验表明, 相对性原理是有效的.
			- 实验表明, 瞬时相互作用在自然界不存在, 相互作用的传播需要时间.
			- 1881年Michelson-Morley干涉实验, 表明光速与其传播方向无关. 而按经典力学,光应在地球速度同方向$(v+c)$,比反方向$(v-c)$更快一点. 因此, Galilean变换的绝对时间假设$(t=t')$错了, 不同参考系下时间流逝的速度不同.
	* Minkowski时空
		\def
			Minkowski时空是三维欧几里得空间和时间的组合四维流形$(t, x, y, z)$.
		* 事件
			\def
				一个事件由其发生的位置和时间所描述$(t, x, y, z)$.
		* 事件间隔
			\def
				$
					s_{12} = (c^2 (t_2-t_1)^2 - (x_2-x_1)^2 - (y_2-y_1)^2 - (z_2-z_1)^2)^{\frac{1}{2}}
					\d s = (c^2 \d t^2 - \d x^2 - \d y^2 - \d z^2)^{\frac{1}{2}}
				$
			\Property
				- 两个事件的间隔在任何惯性系下都一样. 这个不变性,就是光速不变的数学表示.				
				- 类时间隔, 仅在时间上有变化的间隔, 故间隔为实数.
				- 类空间隔, 仅在空间上有变化的间隔, 故间隔为虚数.
	* 参考系变换 -- Lorentz变换
		\Theorem{Lorentz变换}
			$
				(\mb t' \\ x' \\ y' \\ z' \me) = (\mb
					\frac{1}{\sqrt{1-(\frac{V}{c})^2}} & \frac{-\frac{V}{c^2}}{\sqrt{1-( \frac{V}{c} )^2}}
					\frac{-V}{\sqrt{1-( \frac{V}{c} )^2}} & \frac{1}{\sqrt{1-( \frac{V}{c} )^2}}
					& & 1
					& & & 1
				\me) (\mb t \\ x \\ y \\ z \me)
			$
			设两个惯性参考系 $K, K'$, 其中$K'$沿$x$轴以速度$V$相对于$K$作相对运动.
			\Proof
				- 若新旧两个参考系的相对速度不变, 则参考系变换是一种线性变换.
					$(\mb c t' \\ x' \\ y' \\ z' \me) = \.A (\mb c t \\ x \\ y \\ z \me)$
				- 光速不变原理. 所要求的参考系变换, 能够使得Minkowski时空中, 变换前后所有事件间隔不变.
					$
						Δs^2 &= c^2 Δt^2 - Δx^2 - Δy^2 - Δz^2 
								&= c^2 Δt'^2 - Δx'^2 - Δy'^2 - Δz'^2 
								&= const.
					$
					化为矩阵式, 可解得 $\.\eta = \.A^T \.\eta \.A$.
					$
						=> \quad& (\mb c Δt \\ Δx \\ Δy \\ Δz \me)^T (\mb -1 \\ & 1\\ & & 1\\ & & & 1 \me) (\mb c Δt \\ Δx \\ Δy \\ Δz \me) = (\mb c Δt' \\ Δx' \\ Δy' \\ Δz' \me)^T (\mb -1 \\ & 1\\ & & 1\\ & & & 1 \me) (\mb c Δt' \\ Δx' \\ Δy' \\ Δz' \me) = const.
						=> \quad& \.p^T \.\eta \.p = \.p'^T \.\eta \.p'  \tag{简写}
							& \.p^T \.\eta \.p = (\.A \.p)^T \.\eta (\.A \.p)  \tag{代入}
							& \.p^T \.\eta \.p = \.p^T \.A^T \.\eta \.A \.p  \tag{转置}
						=> \quad& \.\eta = \.A^T \.\eta \.A  \tag{矩阵计算}
					$
				- 能够满足事件间隔不变条件的变换, 只有平移和旋转. 排除平移、三维空间旋转、空间反射、时间反演等已经熟悉的变换, 于是, 我们重点关注的是$(t x, t y, t z)$平面内的旋转. 假设y轴z轴不做变换, 只有t轴x轴有变换, 以简化问题并排除三维空间旋转, 可得,
					$
						=> \quad& (\mb c Δt' \\ Δx' \me) = \.A_{t,x} (\mb c Δt \\ Δx \me)
						& (\mb -1 \\ & 1 \me) = \.A_{t,x} (\mb -1 \\ & 1 \me) \.A_{t,x}
						=> \quad& \.A_{t,x} = (\mb a & b \\ c & d \me) \quad  \{\mb
								-a^2 + c^2 = -1
								-b^2 + d^2 =  1
								-a b + c d =  0
							\me\right.
					$
					可知双曲函数$\cosh^2 x - \sinh^2 x = 1$是一组满足该方程组的解, 这个解即是$t x$平面的旋转变换矩阵,
					$\.A_{t,x} = (\mb \cosh \mu & \sinh \mu \\ \sinh \mu & \cosh \mu \me)$
				- 新惯性参考系中原点坐标恒为零, 而在旧惯性参考系中新系原点以速度$V$沿$x$轴运动, 代入可解出$\mu$和目标变换矩阵.
					$(\mb c t' \\ 0 \me) = \.A_{t,x} (\mb c t \\ V t \me) = (\mb \cosh \mu & \sinh \mu \\ \sinh \mu & \cosh \mu \me) (\mb c t \\ V t \me)$
					$
						=> \quad & \mu = artanh(-\frac{V}{c})
						=> \quad & \{\mb
								\tanh \mu = -\frac{V}{c} = \frac{\sinh \mu}{\cosh \mu}
								\cosh^2 \mu - \sinh^2 \mu = 1
							\me\right.
						=> \quad & \{\mb
								\sinh \mu = \frac{-\frac{V}{c}}{\sqrt{1 - (\frac{V}{c})^2}}
								\cosh \mu = \frac{1}{\sqrt{1 - (\frac{V}{c})^2}}
							\me\right.
						=> \quad & (\mb c t' \\ x' \me) = (\mb
								\frac{1}{\sqrt{1-( \frac{V}{c} )^2}} & \frac{-\frac{V}{c}}{\sqrt{1-(\frac{V}{c} )^2}}\\
								\frac{-\frac{V}{c}}{\sqrt{1-( \frac{V}{c} )^2}} & \frac{1}{\sqrt{1-( \frac{V}{c} )^2}}
							\me) (\mb c t \\ x \me)
						=> \quad & (\mb t' \\ x' \me) = (\mb
								\frac{1}{\sqrt{1-( \frac{V}{c} )^2}} & \frac{-\frac{V}{c^2}}{\sqrt{1-(\frac{V}{c} )^2}}\\
								\frac{-V}{\sqrt{1-( \frac{V}{c} )^2}} & \frac{1}{\sqrt{1-( \frac{V}{c} )^2}}
							\me) (\mb t \\ x \me)
					$
		\Example
			* 钟慢效应
			* 尺缩效应
		* 速度变换
			\Theorem{Lorentz变换下的速度变换}
				$ => v_x = \frac{v'_x + V}{1 + v'_x \frac{V}{c^2}}, \quad v_y = \frac{v'_y \sqrt{1 - \frac{V^2}{c^2}}}{1 + v'_x \frac{V}{c^2}},\quad v_z = \frac{v'_z \sqrt{1 - \frac{V^2}{c^2}}}{1 + v'_x \frac{V}{c^2}}$
				\Proof 
					$\.v = \frac{d\.r}{dt},\quad v' = \frac{d\.r'}{dt}$
